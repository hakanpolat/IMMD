\documentclass[conference,a4paper,twocolumn]{IEEEtran}
\usepackage{graphicx}
\graphicspath{{images/}}
\usepackage{ifpdf}
\usepackage{cite}
\ifCLASSINFOpdf
  % \usepackage[pdftex]{graphicx}
  % declare the path(s) where your graphic files are
  % \graphicspath{{../pdf/}{../jpeg/}}
  % and their extensions so you won't have to specify these with
  % every instance of \includegraphics
  % \DeclareGraphicsExtensions{.pdf,.jpeg,.png}
\else
  % or other class option (dvipsone, dvipdf, if not using dvips). graphicx
  % will default to the driver specified in the system graphics.cfg if no
  % driver is specified.
  % \usepackage[dvips]{graphicx}
  % declare the path(s) where your graphic files are
  % \graphicspath{{../eps/}}
  % and their extensions so you won't have to specify these with
  % every instance of \includegraphics
  % \DeclareGraphicsExtensions{.eps}
\fi

\usepackage[cmex10]{amsmath}
\interdisplaylinepenalty=2500

\usepackage{algorithmic}
\usepackage{array}
\usepackage{mdwmath}
\usepackage{mdwtab}
\usepackage{eqparbox}
\usepackage[tight,footnotesize]{subfigure}
\usepackage[caption=false]{caption}
\usepackage[font=footnotesize]{subfig}
\usepackage{fixltx2e}
\usepackage{color}
\usepackage{mathtools}

%\usepackage{stfloats}
% stfloats.sty was written by Sigitas Tolusis. This package gives LaTeX2e
% the ability to do double column floats at the bottom of the page as well
% as the top. (e.g., "\begin{figure*}[!b]" is not normally possible in
% LaTeX2e). It also provides a command:
%\fnbelowfloat
% to enable the placement of footnotes below bottom floats (the standard
% LaTeX2e kernel puts them above bottom floats). This is an invasive package
% which rewrites many portions of the LaTeX2e float routines. It may not work
% with other packages that modify the LaTeX2e float routines. The latest
% version and documentation can be obtained at:
% http://www.ctan.org/tex-archive/macros/latex/contrib/sttools/
% Documentation is contained in the stfloats.sty comments as well as in the
% presfull.pdf file. Do not use the stfloats baselinefloat ability as IEEE
% does not allow \baselineskip to stretch. Authors submitting work to the
% IEEE should note that IEEE rarely uses double column equations and
% that authors should try to avoid such use. Do not be tempted to use the
% cuted.sty or midfloat.sty packages (also by Sigitas Tolusis) as IEEE does
% not format its papers in such ways.

% *** PDF, URL AND HYPERLINK PACKAGES ***
%
%\usepackage{url}
% url.sty was written by Donald Arseneau. It provides better support for
% handling and breaking URLs. url.sty is already installed on most LaTeX
% systems. The latest version can be obtained at:
% http://www.ctan.org/tex-archive/macros/latex/contrib/misc/
% Read the url.sty source comments for usage information. Basically,
% \url{my_url_here}.

% *** Do not adjust lengths that control margins, column widths, etc. ***
% *** Do not use packages that alter fonts (such as pslatex).         ***
% There should be no need to do such things with IEEEtran.cls V1.6 and later.
% (Unless specifically asked to do so by the journal or conference you plan
% to submit to, of course. )


% correct bad hyphenation here
\hyphenation{op-tical net-works semi-conduc-tor}


\begin{document}
\title{Elimination of the DC Bus Sixth Harmonic Component in Integrated Modular Motor Drives Using Third Harmonic Injection Method}
\author{\IEEEauthorblockN{Mesut Uğur}
\IEEEauthorblockA{Department of Electrical and Electronics Engineering\\
Middle East Technical University\\
Ankara, Turkey\\
Email: ugurm@metu.edu.tr}
\and
\IEEEauthorblockN{Ozan Keysan}
\IEEEauthorblockA{Department of Electrical and Electronics Engineering\\
Middle East Technical University\\
Ankara, Turkey\\
Email: keysan@metu.edu.tr}
}

\maketitle

\begin{abstract}
%\boldmath
In this paper, a novel method to eliminate the harmonic component occurring on the DC bus which is six times the grid frequency is proposed. This harmonic component is present due to natural commutation of the passive diode bridge rectifier in motor drive applications. In conventional drives, bulky LC filters are utilized to reduce the effect of this harmonic component to the motor drive inverter. With this method, DC bus capacitance requirement can be minimized which will enhance the power density and decrease the cost of the overall system. Third harmonic injection is used with modular inverters in an integrated modular motor drive application. Both rectifier and inverter side analytical models are presented, the elimination of the sixth harmonic component is described analytically, and verified by simulations performed on MATLAB/Simulink. The possible adverse effects of third harmonic injection method are also discussed.
\end{abstract}
\IEEEpeerreviewmaketitle



\section{Introduction}
This demo file is intended to serve as a ``starter file''
for IEEE conference papers produced under \LaTeX\ using
IEEEtran.cls version 1.7 and later.

\hfill mds
 
\hfill January 11, 2007

\subsection{Subsection Heading Here}
Subsection text here.

\subsubsection{Subsubsection Heading Here}
Subsubsection text here.



%\begin{figure}[!t]
%\centering
%\includegraphics[width=8cm]{images/conv_motor_drive}
%\caption{A conventional motor drive block diagram}
%\label{fig:conv_motor_drive}
%\end{figure}


%\begin{figure*}[!t]
%\centerline{\subfloat[Case I]\includegraphics[width=2.5in]{subfigcase1}%
%\label{fig_first_case}}
%\hfil
%\subfloat[Case II]{\includegraphics[width=2.5in]{subfigcase2}%
%\label{fig_second_case}}}
%\caption{Simulation results}
%\label{fig_sim}
%\end{figure*}


% The following is an example of commenting a large section
\iffalse
\begin{table}[htp]
\renewcommand{\arraystretch}{1.3}
\caption{An Example of a Table}
\label{table_example}
\centering
\begin{tabular}{|c|c|c|}
\hline
One & Two & a\\
\hline
Three & Four & a\\
\hline
Three & Four & b\\
\hline
\end{tabular}
\end{table}
\fi

Most studies consider only one side for DC link characterisation or filter component optimization, although they should be considered simultaneously. This research aims at modeling the system as a whole, investigating the effect of harmonic components injected to the DC link from both sides and eliminating the low frequency harmonic due to the rectifier side by using the modular structure of the inverter side.

\section{Problem definition}
%analytical Rectifier model, 6th harmonic component %injection, LC filter characteristics

A conventional motor drive system block diagram is shown in Fig. \ref{fig:conv_motor_drive}. The rectifier and inverter are connected via the DC link, therefore a harmonic component injected from one side may be reflected to the other side, disrupting its operation. For systems having two active converters on both sides such as back-to-back converters, the only fluctuations seen on the DC link voltage are high frequency components which are directly related to the switching frequencies of each side. On the other hand, in case of passive converters such as diode bridge rectifiers, low frequency components emerge on the DC link voltage which are related to the grid supply frequency. Since grid frequency is usually much smaller than the switching frequencies applied to active converters, filtering of fluctuations on the DC link requires much larger and more expensive filter components in case of passive rectifiers.

\begin{figure}[htp]
  \centering
  \includegraphics[width=8cm]{images/conv_motor_drive}
  \caption{A conventional motor drive system block diagram}
  \label{fig:conv_motor_drive}
\end{figure}

The diode bridge rectifier is a natural-commutated converter, circuit schematic of which is shown in Fig. \ref{fig:rect_circuit}. A second order LC filter which is of low-pass type is usually utilized at the rectifier output for filtering. In this section, the harmonic content of the rectifier output voltage and the effect of filtering are shown analytically.

\begin{figure}[htp]
  \centering
  \includegraphics[width=8cm]{images/rect_circuit}
  \caption{Diode bridge rectifier circuit diagram}
  \label{fig:rect_circuit}
\end{figure}

The supply voltages are shown in \ref{eq:1}-\ref{eq:3}, where $V_m$ is the RMS value of the supply voltage and $f_s$ is the supply frequency. With this configuration, integer multiples of the harmonic component frequency of which is six times the grid frequency $f_s$ are present on the rectifier output voltage in addition to the DC component as shown in \ref{eq:4}. 

\begin{equation} \label{eq:1}
v_{sa}(t) = \sqrt{2}V_msin(2\pi f_s t)
\end{equation}
\begin{equation} \label{eq:2}
v_{sb}(t) = \sqrt{2}V_msin(2\pi f_s t-2\pi/3)
\end{equation}
\begin{equation} \label{eq:3}
v_{sc}(t) = \sqrt{2}V_msin(2\pi f_s t-4\pi/3)
\end{equation}
\begin{equation} \label{eq:4}
v_{dc}(t) = \frac{3\sqrt{3}}{\pi} \bigg[1-\sum_{k=1}^{\infty} \frac{2}{36k^2-1}cos(6k\omega_{0}t)\bigg] 
%\sqrt{2}V_msin(2\pi f_s t)
\end{equation}

The transfer function between $v_load$ and $v_dc$ is obtained to show the characteristics of the filter as in \ref{eq:5}. Magnitude and phase of the harmonic components that are of interest are obtained from this model and they are later used in the implementation of the proposed method.

\begin{equation} \label{eq:5}
H(jw) = \frac{v_{L}(jw)}{v_{dc}(jw)} = \frac{R_{L}}{R_{L}(1-w^2L_{dc}C_{dc})+jwL_{dc}}
\end{equation}

A simulation is performed for the verification of these analytical models on MATLAB/Simulink, for 400V line-to-line grid voltage at 50 Hz, filter inductance of 1 mH, filter capacitance of 4 mF and load resistance of 10 $\Omega$. In Fig. \ref{fig:rect_wave2}, the three-phase input voltages, the rectifier output voltage and the filtered load voltage waveforms are shown. In Fig. \ref{fig:rect_harm1} and \ref{fig:rect_harm2}, the harmonic spectra of rectifier output voltage and load voltage are shown, respectively.

\begin{figure}[htp]
  \centering
  \includegraphics[width=8cm]{images/rect_wave2}
  \caption{Diode bridge rectifier input and output waveforms}
  \label{fig:rect_wave2}
\end{figure}

\begin{figure}[htp]
  \centering
  \includegraphics[width=8cm]{images/rect_harm1}
  \caption{Harmonic spectrum of rectifier output voltage}
  \label{fig:rect_harm1}
\end{figure}

\begin{figure}[htp]
  \centering
  \includegraphics[width=8cm]{images/rect_harm2}
  \caption{Harmonic spectrum of load voltage}
  \label{fig:rect_harm2}
\end{figure}

A load voltage peak-to-peak ripple of 1\% is usually aimed in case of motor drive inverter applications and the filter values are adjusted here such that a ripple of \textbf{X}\% is obtained. The inductance value is usually determined by the current ripple on this inductor and it has minor effect on the load voltage ripple. Capacitance values on the DC link are usually in a few hundred microfarad range when only high frequency fluctuations are considered, however in this case, capacitance values in milifarad range are needed.

In IMMD applications, decreasing the volume of the passive elements is a major challenge due to having small volume. Therefore, integration of the motor drive with passive rectifiers to the motor is a problem.


\textbf{\textcolor{red}{buraya \% 30unu kaplıyor gibi istatistikler gelebilir.}}


\section{Description of the proposed method}
In the proposed method, a harmonic component which is six times the grid frequency is aimed to be created at the inverter DC input such that there will be no low order harmonic current flowing through the DC link capacitor. By doing so, the DC link capacitance requirement can be reduced significantly.

The harmonic component is created by injecting zero sequence third harmonic components at the motor side. Voltages and current expressions of one inverter module with zero sequence third harmonic injection are shown in \ref{eq:1}-\ref{eq:6}.

\begin{equation} \label{eq:1}
v_a(t) = V_1sin(2\pi ft-\phi_{1v})+V_3sin(6\pi ft-\phi_{3v})
\end{equation}
\begin{equation} \label{eq:2}
v_b(t) = V_1sin(2\pi ft-2\pi/3-\phi_{1v})+V_3sin(6\pi ft-\phi_{3v})
\end{equation}
\begin{equation} \label{eq:3}
v_c(t) = V_1sin(2\pi ft-4\pi/3-\phi_{1v})+V_3sin(6\pi ft-\phi_{3v})
\end{equation}
\begin{equation} \label{eq:4}
i_a(t) = I_1sin(2\pi ft-\phi_{1i})+I_3sin(6\pi ft-\phi_{3i})
\end{equation}
\begin{equation} \label{eq:5}
i_b(t) = I_1sin(2\pi ft-2\pi/3-\phi_{1i})+I_3sin(6\pi ft-\phi_{3i})
\end{equation}
\begin{equation} \label{eq:6}
i_c(t) = I_1sin(2\pi ft-4\pi/3-\phi_{1i})+I_3sin(6\pi ft-\phi_{3i})
\end{equation}

Let us make the definitions shown in \ref{eq:7}-\ref{eq:14}.

\begin{equation} \label{eq:7}
\phi_{11p} = \phi_{1v}+\phi_{1i}
\end{equation}
\begin{equation} \label{eq:8}
\phi_{33p} = \phi_{3v}+\phi_{3i}
\end{equation}
\begin{equation} \label{eq:9}
\phi_{13p} = \phi_{1v}+\phi_{3i}
\end{equation}
\begin{equation} \label{eq:10}
\phi_{31p} = \phi_{3v}+\phi_{1i}
\end{equation}
\begin{equation} \label{eq:11}
\phi_{11n} = \phi_{1v}-\phi_{1i}
\end{equation}
\begin{equation} \label{eq:12}
\phi_{33n} = \phi_{3v}-\phi_{3i}
\end{equation}
\begin{equation} \label{eq:13}
\phi_{13n} = \phi_{1v}-\phi_{3i}
\end{equation}
\begin{equation} \label{eq:14}
\phi_{31n} = \phi_{3v}-\phi_{1i}
\end{equation}

The instantaneous power expression for each phase are shown in \ref{eq:15}-\ref{eq:17}.

\begin{equation}
\label{eq:15}
\begin{multlined}
p_a(t) = 
\frac{V_1I_1}{2} \bigg \lbrack cos(\phi_{11n})-cos(4\pi ft-\phi_{11p}) \bigg \rbrack
\\
+
\frac{V_1I_3}{2} \bigg \lbrack cos(4\pi ft+\phi_{13n})-cos(8\pi ft-\phi_{13p}) \bigg \rbrack
\\
+
\frac{V_3I_1}{2} \bigg \lbrack cos(4\pi ft-\phi_{31n})-cos(8\pi ft-\phi_{31p}) \bigg \rbrack
\\
+
\frac{V_3I_3}{2} \bigg \lbrack cos(\phi_{33n})-cos(12\pi ft-\phi_{33p}) \bigg \rbrack,
\end{multlined}
\end{equation}

\begin{equation}
\label{eq:16}
\begin{multlined}
p_b(t) = 
\frac{V_1I_1}{2} \bigg \lbrack cos(\phi_{11n})-cos(4\pi ft-4\pi/3-\phi_{11p}) \bigg \rbrack
\\
+
\frac{V_1I_3}{2} \bigg \lbrack cos(4\pi ft+ 2\pi/3+\phi_{13n})-cos(8\pi ft-2\pi/3-\phi_{13p}) \bigg \rbrack
\\
+
\frac{V_3I_1}{2} \bigg \lbrack cos(4\pi ft+2\pi/3-\phi_{31n})-cos(8\pi ft-2\pi/3-\phi_{31p}) \bigg \rbrack
\\
+
\frac{V_3I_3}{2} \bigg \lbrack cos(\phi_{33n})-cos(12\pi ft-\phi_{33p}) \bigg \rbrack,
\end{multlined}
\end{equation}

\begin{equation}
\label{eq:17}
\begin{multlined}
p_c(t) = 
\frac{V_1I_1}{2} \bigg \lbrack cos(\phi_{11n})-cos(4\pi ft-8\pi/3-\phi_{11p}) \bigg \rbrack
\\
+
\frac{V_1I_3}{2} \bigg \lbrack cos(4\pi ft+ 4\pi/3+\phi_{13n})-cos(8\pi ft-4\pi/3-\phi_{13p}) \bigg \rbrack
\\
+
\frac{V_3I_1}{2} \bigg \lbrack cos(4\pi ft+4\pi/3-\phi_{31n})-cos(8\pi ft-4\pi/3-\phi_{31p}) \bigg \rbrack
\\
+
\frac{V_3I_3}{2} \bigg \lbrack cos(\phi_{33n})-cos(12\pi ft-\phi_{33p}) \bigg \rbrack,
\end{multlined}
\end{equation}

The total instantaneous power becomes as in \ref{eq:18}. As seen, all the frequency components which are two times and four times the fundamental frequency are cancelled, leaving two DC components and a component at six times the fundamental frequency. The last term will be used in this method for the cancellation of the sixth harmonic injected by the rectifier.

\begin{equation}
\label{eq:18}
\begin{multlined}
p_{total}(t) = 
\frac{V_1I_1}{2} \bigg \lbrack cos(\phi_{11n}) \bigg \rbrack
+
\frac{V_3I_3}{2} \bigg \lbrack cos(\phi_{33n}) \bigg \rbrack
\\
+
\frac{V_3I_3}{2} \bigg \lbrack cos(12\pi ft-\phi_{33p}) \bigg \rbrack
\end{multlined}
\end{equation}


\section{Implementation of the method and practical issues}

This method is proposed for IMMD applications where selection of  number of inverter modules, number of stator phases etc. are flexible. This makes the application of the method more convenient. In this paper, a six pole machine is considered. The schematic of the segmented stator and modular inverters is shown in Fig. \ref{fig:rect_circuit}. 

\begin{figure}[htp]
  \centering
  \includegraphics[width=8cm]{images/rect_circuit}
  \caption{IMMD scheme}
  \label{fig:rect_circuit}
\end{figure}

It has been shown that, with conventional three-phase connection, injection of third harmonic is impossible, regardless of the winding connection being delta or wye, due to the nature of the three-phase. Therefore, in this paper, an IMMD scheme is proposed such that each stator coil is fed by a separate single-phase bridge inverter.

The conventional connection and the proposed connection are shown in Fig. \ref{fig:rect_circuit}.

\begin{figure}[htp]
  \centering
  \includegraphics[width=8cm]{images/rect_circuit}
  \caption{Conventional connection and proposed connection}
  \label{fig:rect_circuit}
\end{figure}

The drawbacks of such a connection are as follows: bunları açıkla

\begin{enumerate}
  \item Increased number of switches: cost, açıkla
  \item Conduction loss, switching loss
  \item Copper loss, core loss
  \item Torque ripple
\end{enumerate}

Additional advantages: bunları açıkla
\begin{enumerate}
  \item Increased modularity
  \item Conduction loss, switching loss
\end{enumerate}

New IMMD scheme for third harmonic injection, practical considerations

Give the block diagram and the modular structure with split windings at a separate diagram

DC linkte paralel bağlandığını söyle

Effect on torque ripple
The discussion on the elimination of the harmonic when there is no third space harmonic
A method can be developed to eliminate the third space harmonic


\section{Results}
Simulation results

Give the parameters, specifications
Total output power
Switching frequency
Motor fundamental frequency
Supply voltage, frequency
Modulation index
Number of modules
Number of series and parallel connected modules
Give the inductance capacitance, motor parameters etc.

1. The rectifier output current, the inverter input current, and DC link capacitor current at the same figure

2. DC link capacitor current in a separate figure
The percent decrease of the sixth harmonic current on the DC link capacitor, percent decrease of the overall capacitor RMS current

3. The motor line currents, transistor currents
The increase on the conduction losses of the switches and the copper losses of the windings

4. Current of the inverters separately, show how the sixth harmonic is created

5. Discuss the variable frequency case

6. Discuss the capacitor requirement
There is no effect on the capacitance, but the RMS current rating decreases
This will have no effect on film capacitors unless switching frequency is higher than 150 kHz
But, the requirement on the RMS current will decrease significantly so that aluminium electrolytic capacitors can be made smaller now
This will have the effects of:
- Lower cost
- Higher power density
- Lifetime may be extended


\section{Conclusion}
The conclusion goes here.


\section*{Acknowledgment}

The authors would like to thank...



\begin{thebibliography}{1}

\bibitem{IEEEhowto:kopka}
H.~Kopka and P.~W. Daly, \emph{A Guide to \LaTeX}, 3rd~ed.\hskip 1em plus
  0.5em minus 0.4em\relax Harlow, England: Addison-Wesley, 1999.

\end{thebibliography}


\end{document}


