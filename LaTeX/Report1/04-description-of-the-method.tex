\chapter{Description of the Method}
\addcontentsline{chapter}{Description of the Method}

Voltages and current expressions of one inverter module with zero sequence third harmonic injection are shown in \ref{eq:1}-\ref{eq:6}.

\begin{equation} \label{eq:1}
v_a(t) = V_1sin(2\pi ft-\phi_{1v})+V_3sin(6\pi ft-\phi_{3v})
\end{equation}
\begin{equation} \label{eq:2}
v_b(t) = V_1sin(2\pi ft-2\pi/3-\phi_{1v})+V_3sin(6\pi ft-\phi_{3v})
\end{equation}
\begin{equation} \label{eq:3}
v_c(t) = V_1sin(2\pi ft-4\pi/3-\phi_{1v})+V_3sin(6\pi ft-\phi_{3v})
\end{equation}

\begin{equation} \label{eq:4}
i_a(t) = I_1sin(2\pi ft-\phi_{1i})+I_3sin(6\pi ft-\phi_{3i})
\end{equation}
\begin{equation} \label{eq:5}
i_b(t) = I_1sin(2\pi ft-2\pi/3-\phi_{1i})+I_3sin(6\pi ft-\phi_{3i})
\end{equation}
\begin{equation} \label{eq:6}
i_c(t) = I_1sin(2\pi ft-4\pi/3-\phi_{1i})+I_3sin(6\pi ft-\phi_{3i})
\end{equation}

Let us make the definitions shown in \ref{eq:7}-\ref{eq:14}.

\begin{equation} \label{eq:7}
\phi_{11p} = \phi_{1v}+\phi_{1i}
\end{equation}
\begin{equation} \label{eq:8}
\phi_{33p} = \phi_{3v}+\phi_{3i}
\end{equation}
\begin{equation} \label{eq:9}
\phi_{13p} = \phi_{1v}+\phi_{3i}
\end{equation}
\begin{equation} \label{eq:10}
\phi_{31p} = \phi_{3v}+\phi_{1i}
\end{equation}
\begin{equation} \label{eq:11}
\phi_{11n} = \phi_{1v}-\phi_{1i}
\end{equation}
\begin{equation} \label{eq:12}
\phi_{33n} = \phi_{3v}-\phi_{3i}
\end{equation}
\begin{equation} \label{eq:13}
\phi_{13n} = \phi_{1v}-\phi_{3i}
\end{equation}
\begin{equation} \label{eq:14}
\phi_{31n} = \phi_{3v}-\phi_{1i}
\end{equation}

The instantaneous power expression for each phase are shown in \ref{eq:15}-\ref{eq:17}.

\begin{equation}
\label{eq:15}
\begin{multlined}
p_a(t) = 
\frac{V_1I_1}{2} \bigg \lbrack cos(\phi_{11n})-cos(4\pi ft-\phi_{11p}) \bigg \rbrack
+
\frac{V_1I_3}{2} \bigg \lbrack cos(4\pi ft+\phi_{13n})-cos(8\pi ft-\phi_{13p}) \bigg \rbrack
\\
+
\frac{V_3I_1}{2} \bigg \lbrack cos(4\pi ft-\phi_{31n})-cos(8\pi ft-\phi_{31p}) \bigg \rbrack
+
\frac{V_3I_3}{2} \bigg \lbrack cos(\phi_{33n})-cos(12\pi ft-\phi_{33p}) \bigg \rbrack,
\end{multlined}
\end{equation}

\begin{equation}
\label{eq:16}
\begin{multlined}
p_b(t) = 
\frac{V_1I_1}{2} \bigg \lbrack cos(\phi_{11n})-cos(4\pi ft-4\pi/3-\phi_{11p}) \bigg \rbrack
\\
+
\frac{V_1I_3}{2} \bigg \lbrack cos(4\pi ft+ 2\pi/3+\phi_{13n})-cos(8\pi ft-2\pi/3-\phi_{13p}) \bigg \rbrack
\\
+
\frac{V_3I_1}{2} \bigg \lbrack cos(4\pi ft+2\pi/3-\phi_{31n})-cos(8\pi ft-2\pi/3-\phi_{31p}) \bigg \rbrack
\\
+
\frac{V_3I_3}{2} \bigg \lbrack cos(\phi_{33n})-cos(12\pi ft-\phi_{33p}) \bigg \rbrack,
\end{multlined}
\end{equation}

\begin{equation}
\label{eq:17}
\begin{multlined}
p_c(t) = 
\frac{V_1I_1}{2} \bigg \lbrack cos(\phi_{11n})-cos(4\pi ft-8\pi/3-\phi_{11p}) \bigg \rbrack
\\
+
\frac{V_1I_3}{2} \bigg \lbrack cos(4\pi ft+ 4\pi/3+\phi_{13n})-cos(8\pi ft-4\pi/3-\phi_{13p}) \bigg \rbrack
\\
+
\frac{V_3I_1}{2} \bigg \lbrack cos(4\pi ft+4\pi/3-\phi_{31n})-cos(8\pi ft-4\pi/3-\phi_{31p}) \bigg \rbrack
\\
+
\frac{V_3I_3}{2} \bigg \lbrack cos(\phi_{33n})-cos(12\pi ft-\phi_{33p}) \bigg \rbrack,
\end{multlined}
\end{equation}

The total instantaneous power becomes as in \ref{eq:18}. As seen, all the frequency components which are two times and four times the fundamental frequency are cancelled, leaving two DC components and a component at six times the fundamental frequency.

\begin{equation}
\label{eq:18}
\begin{multlined}
p_{total}(t) = 
\frac{V_1I_1}{2} \bigg \lbrack cos(\phi_{11n}) \bigg \rbrack
+
\frac{V_3I_3}{2} \bigg \lbrack cos(\phi_{33n}) \bigg \rbrack
+
\frac{V_3I_3}{2} \bigg \lbrack cos(12\pi ft-\phi_{33p}) \bigg \rbrack
\end{multlined}
\end{equation}

